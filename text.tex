\documentclass{article}
\usepackage[utf8]{inputenc}
\usepackage[portuguese]{babel}
\usepackage{amsmath}
\usepackage{physics}
\usepackage{amsmath}
\usepackage{tikz}
\usepackage{mathdots}
\usepackage{yhmath}
\usepackage{cancel}
\usepackage{color}
\usepackage{siunitx}
\usepackage{array}
\usepackage{multirow}
\usepackage{amssymb}
\usepackage{gensymb}
\usepackage{tabularx}
\usepackage{booktabs}

\usetikzlibrary{patterns}
\usetikzlibrary{shadows.blur}
\usetikzlibrary{shapes}

\title{Trabalho Final - Segmentação de Imagem Utilizando Grafos}
\author{Larissa Domingues Gomes - 650525 \\
Pedro Henrique Lima Carvalho - 651230\\
Tárcila Fernanda Resende da Silva - 680250}
\date{Junho 2021}

\usepackage{natbib}
\usepackage{graphicx}

\begin{document}

\maketitle

\section{Introdução}
\quad Uma das principais aplicações de algoritmos em grafos é o processamento de imagem. Neste trabalho será discorrido sobre a segmentação de imagem utilizando um algoritmo de fluxo em redes.\\

\section{Construção do Grafo a partir de uma Imagem}
\quad A função \textbf{buildGraph} é responsável por construir um grafo a partir de um arquivo de imagem \textbf{.pgm}. Inicialmente, o método abre o arquivo recebido como parâmentro e lê o cabeçalho com os respectivos valores de largura, altura e valor de cinza máximo da imagem.\\

 Para a construção do grafo, são criados vértices para cada pixel da imagem e dois vértices a mais para serem os respectivos source e sink utilizados no algoritmo de fluxo em redes. A posição dos pixels de referência para o source e sink são recebidos como parâmetro. Cada vértice de pixel armazena o valor númerico da cor.\\
 
 Os vértices de pixel são ligados por uma aresta valorada aos seus respectivos vizinhos. O valor armazenado em cada aresta é o seu fluxo, quanto maior o valor do fluxo, mais semelhante é a cor do pixel vizinhos. Existe um outro tipo de vértice que conecta o source e o sink aos pixels que têm  tonalidades parecidas.
 
 \section{Segmentação da Imagem}
\quad A função \textbf{maxFlow} retorna o fluxo máximo do grafo da imagem utilizando o algoritmo de Ford-Fulkerson. A partir do grafo residual de fluxo máximo, é possível encontrar o corte, que representa a separação do fundo com o primeiro plano da imagem.\\

A ideia principal do algoritmo de segmentação é aumentar o fluxo de entrada entre cores que se parecem, desta forma os pixels que são parecidos irão fazer parte da separação do seu respectivo plano. A saída contem uma imagem com pixels pretos para representar o primeiro plano da imagem e brancos para pixels do fundo.

\end{document}

